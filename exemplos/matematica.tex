%
% matematica.tex
%
% Workshop de LaTeX do SciELO
%
% Demonstra:
% - Ambientes para inserir equações: math, displaymath e equation
% - Como inserir símbolos, letras gregas, operadores, potências e subscritos,
%   frações e raízes
%

\documentclass[a4paper,oneside]{article}
\usepackage{fontspec}
\usepackage{polyglossia}
  \setdefaultlanguage{brazil}
\usepackage{mathtools}

\begin{document}
\frenchspacing

\section{Três ambientes para matemática}

No \LaTeX, há três ambientes para acessar o modo de matemática. Eles são o
\texttt{math}, \texttt{displaymath} e \texttt{equation}. O primeiro é do tipo
\emph{inline} (o ambiente não cria um novo parágrafo), enquanto que os dois
últimos são do tipo \emph{displayed} (o ambiente cria um novo parágrafo).
É possível acessar o ambiente \texttt{math} usando o atalho \verb+\( … \)+ e o
ambiente \texttt{displaymath} é equivalente a \verb+\[ … \]+.

% Mostrar o efeito dos três ambientes discutidos acima, além dos comandos
% \label e \tag.
\bigskip
Newton demostrou que a força gravitacional entre dois objetos é igual a \( F =
G \frac{m_1 m_2}{r^2} \) em 1687, quando publicou o \emph{Principia}.

\section{Símbolos}

Há centenas de símbolos já implementados no LaTeX, além daqueles que você tem
acesso no seu teclado. Aqui estamos combinando os dois tipos:

\[ 2 \times 2 = 4 \]

No modo de matemática, o espaçamento funciona de maneira diferente. O LaTeX
automaticamente escolhe o espaçamento entre os símbolos, números etc. Por
exemplo:

\[ ab = ba \]

\section{Letras gregas}

É fácil acessar letras gregas:

\[ \text{Circunferência} = 2 \pi r \]

\section{Operadores}

De maneira similar, muitos operadores já estão implementados: \( \log xy = \log
x + \log y \).

Uma das formas de criptografia mais antigas é conhecida como a Cifra de César:

\begin{equation}
  E_n(x) = (x + n) \bmod 26
  \tag{Cifra de César}
\end{equation}

\section{Potências e subscritos}

Potências são representadas com acentos circunflexos, \( 2^8 \). Subscritos são
representados com underlines, \( a_b \). Como em muitos outros casos no modo de
matemática, é possível agrupar valores usando chaves: \( 2^{32} \).

\[ f(n) = 4n + n^2 \]

\section{Frações}

Frações podem ser usadas com o comando \verb+\frac+: \( \frac{2}{5} \). Também
é possível incluir frações dentro de frações:

\[ \frac{\frac{1}{x}+\frac{1}{y}}{y-z} \]

\section{Raízes}

Assim como no caso das frações, raízes têm um comando especial: \verb+\sqrt+.
A proporção \( 1:\sqrt{2} \) é usada frequentemente em tipografia. Também é
possível escrever raízes com expoentes diferentes:

\[ \sqrt[3]{27} = 3 \]
\end{document}
