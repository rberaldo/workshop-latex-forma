%
% tabelas.tex
%
% Workshop de LaTeX
%
% Demonstra:
% - O ambiente tabular
% - Como implementar tabelas usando mais espaço em branco ao invés de linhas
% - Como colocar parágrafos dentro de células
% - O pacote booktabs
% - Células com várias colunas usando \multicolumn
% - O pacote longtable
% - Floats e o ambiente table
% - Posições com o ambiente table
%

\documentclass[a4paper,oneside]{article}
\usepackage{fontspec}
\usepackage{polyglossia}
  \setdefaultlanguage{brazil}
% \usepackage{booktabs}
% \usepackage{longtable}

\begin{document}
\frenchspacing

\section{Uma tabela básica}

Usando a tabela abaixo, vamos aprender como alinhar texto dentro das células de
um ambiente \texttt{tabular}:

\begin{center}
  \begin{tabular}{l c r}
    1 & 2 & 3\\
    4 & 5 & 6\\
    7 & 8 & 9\\
  \end{tabular}
\end{center}

% A seção a seguir foi inspirada no site
% http://www.thefreshloaf.com/lessons/yourfirstloaf
\section{Ingredientes básicos para o pão}

Apenas quatro ingredientes são necessários para fazer pão. Vejamos uma receita
genérica:

% Explorar as possibilidades de alinhamento; demonstrar o uso do ambiente table
\begin{center}
  \begin{tabular}{lr}
    Ingrediente & Quantidade\\[5pt]
    Trigo    & 3 xíc\\
    Sal      & 2 cc\\
    Fermento & 2 cc\\
    Água     & 1 $^1/_7$ xíc
  \end{tabular}
\end{center}

Antes de correr para a cozinha, é fundamental entender para que servem e como
interagem esses ingredientes.

% Trocar o tamanho do parágrafo
\begin{center}
  \begin{tabular}{lp{.6\textwidth}}
      Ingrediente & Função\\[5pt]
      Trigo       & A base do pão. Sem trigo, sem pão.\\
      Sal         & Retarda o fermento e controla o processo de fermentação.
                      Também adiciona o sabor que todo esperamos do pão.\\
      Fermento    & É comumente vendido seco e precisa ser ativado com água
                      morna.  Faz o pão crescer e quanto maior a quantidade,
                      mais rápido é o crescimento. Fermento demais deixa o pão
                      com um gosto de cerveja.\\
      Água        & Dissolve os ingredientes (solvente universal, alguém?) e
                      ativa o fermento. Ao adicionar mais água, o resultado é
                      um pão mais pegajoso e com buracos menos regulares. Se
                      houver muito pouca água, a expansão da massa é restrita e
                      o resultado é um pão mais firme, seco e duro.
  \end{tabular}
\end{center}

Imprima o resumo abaixo e cole na parede da cozinha, para nunca esquecer esses
princípios básicos:

% É possível induzir quebras de linha explícitas dentro de um parágrafo em uma
% célula, se ela incluir o comando \raggedright ou \centering. Nesses casos,
% precisamos usar \tabularnewline para começar uma nova linha.
%
% Também introduzir o pacote booktabs, que nos permite usar os comandos
% \toprule, \midrule and \bottomrule. Discutir como, frequentemente, não são
% necessários para melhorar a legibilidade.
\begin{center}
  \begin{tabular}{lp{.6\textwidth}}
    Ingrediente & Função\tabularnewline[8pt]
    Trigo & Base do pão\tabularnewline[3pt]
    Sal & \raggedright Retarda o fermento\\
                    Controla o processo de fermentação\\
                    Dá sabor ao pão\tabularnewline[3pt]
    Fermento & \raggedright Vendido seco\\
                    Ativado com água morna\\
                    Faz o pão crescer\\
                    Em abundância, o crescimento é mais
                    rápido\tabularnewline[3pt]
    Água & \raggedright Dissolve os ingredientes\\
                    Ativa o fermento\\
                    Em abundância, pão mais pegajoso e com buracos menos
                    regulares\\
                    Em falta, pão mais firme, duro e seco
  \end{tabular}
\end{center}

% Informação encontrada em
% http://www.space.com/images/i/000/024/511/original/nearest-stars-121218g-02.jpg
\section{Distâncias cósmicas}

Com uma nave que viaje à velocidade da luz, seriam necessários 4,2 anos para
chegar à estrela mais próxima. Abaixo, uma lista de destinos para quem dispõe
de tempo livre.

% Estamos demonstrando dois conceitos aqui: como poderíamos usar um \midrule
% depois de multicolumn e, é claro, multicolumn. Vamos aproveitar para
% deixá-la mais longa e carregar o pacote longtable. O comando \endhead é útil
% para definir um cabeçalho que se repete em toda página.
\begin{center}
  \begin{tabular}{lr}
    \multicolumn{2}{c}{Estrelas na Via Láctea}\\[8pt]
    Estrela            & Distância (anos-luz)\\[5pt]
    Proxima Centauri   & 4,2\\
    Rigil Kentaurus    & 4,3\\
    Alpha Cen B        & 4,3\\
    Estrela de Barnard & 6,0\\
    Wolf 359           & 7,7\\
    BD +36 2147        & 8,2\\
    Luyten 726-8A      & 8,4\\
    Luyten 726-8B      & 8,4\\
    Sirius A           & 8,6\\
    Sirius B           & 8,6\\
    Ross 154           & 9,4\\
    Ross 248           & 10,4\\
    Epsilon Eri        & 10,8\\
    Ross 128           & 10,9\\
    61 Cyg A           & 11,1\\
    61 Cyg B           & 11,1\\
    Epsilon Ind        & 11,2\\
    BD +43 44 A        & 11,2\\
    BD +43 44 B        & 11,2\\
    Luyten 789-6       & 11,2\\
    Procyon A          & 11,4\\
    Procyon B          & 11,4\\
    BD +59 1915 A      & 11,6\\
    BD +59 1915 B      & 11,6\\
    CoD -36 15693      & 11,7
  \end{tabular}
\end{center}
\end{document}
