%
% trabalho-normatizado.tex
%
% Workshop de LaTeX do SciELO
%
% Demonstra:
% - Como utilizar a classe abnTeX2 para criar uma tese
% - Organização do texto
% - Comandos específicos da classe
% - O comando \input
%

\documentclass[12pt,oneside,a4paper,english,brazil]{abntex2}
% Não carregaremos o pacote polyglossia, uma vez que é carregado por padrão
% pela classe
\usepackage{microtype,graphicx,hyperref,blindtext,fontspec}
% Pacote usado para citações:
\usepackage[alf]{abntex2cite}

% Informações da capa
\titulo{}
\autor{}
\orientador[Orientadora: ]{}
\data{}
\instituicao{}
\tipotrabalho{}

\begin{document}
\selectlanguage{brazil}
\frenchspacing

\pretextual
\imprimircapa
\imprimirfolhaderosto

% Resumos (adaptado do manual do abntex2)
% --- resumo em português ---
\begin{resumo}
  Resumo em português
  \vspace{\onelineskip}
  \noindent
  \textbf{Palavras-chave}: latex. abntex. editoração de texto.
\end{resumo}
% --- resumo em inglês ---
\begin{resumo}[Abstract]
  \selectlanguage{english}
  Abstract in English.
  \vspace{\onelineskip}
  \noindent
  \textbf{Keywords}: latex. abntex. text publishing.
\end{resumo}

% Para que a ToC seja incluída nas bookmarks do PDF
\tableofcontents*
\cleardoublepage

\textual
\cleardoublepage
\input{introducao}
\input{objetivos-justificativa}
\input{metodologia}
\input{desenvolvimento}
\input{conclusao}

\postextual
\bibliography{trabalho-normatizado}
\end{document}
