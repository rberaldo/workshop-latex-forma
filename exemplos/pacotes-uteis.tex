%
% pacotes-uteis.tex
%
% Workshop de LaTeX do SciELO
%
% Demonstra os pacotes:
% - titling
% - sectsty
% - geometry
% - fancyhdr
% - tcolorbox
% - microtype
%

\documentclass[a4paper]{article}
% --- pacotes necessários ---
\usepackage{polyglossia}
  \setdefaultlanguage{brazil}
\usepackage{minted}
\usepackage{xcolor}
\usepackage{hyperref}
\hypersetup{
  colorlinks,
  linkcolor={red!50!black},
  citecolor={blue!50!black},
  urlcolor={blue!80!black}
}
\usepackage{multicol}
\usepackage{microtype}

% --- pacotes estudados ---
% titling: modificar o comando \maketitle
% \usepackage{titling}
% opções
%
% sectsty: modificar a fonte e posição das seções
% \usepackage{sectsty}
% opções
%
% geometry: controla margens e tamanhos de página
% \usepackage{geometry}
% opções
%
% fancyhdr: cabeçalhos e rodapés
% \usepackage{fancyhdr}
% opções
%
% tcolorbox: cria caixas de texto
% \usepackage{tcolorbox}
% opções
%
% microtype: controles tipográficos avançados
\usepackage{microtype}

% --- macros ---
\newcommand{\code}[1]{\mintinline{latex}{#1}}

% --- título ---
\title{Pacotes úteis: uma demonstração}
\author{João da Silva}
\date{14 de junho}

% --- o documento ---
\begin{document}
\frenchspacing

\maketitle

\section{O pacote \code{titling}}

Ao invés de modificar o comando \code{\maketitle}, podemos utilizar o pacote
\href{https://ctan.org/pkg/titling/}{\code{titling}} e seus comandos para
modificar o título de nossos artigos.

Os principais comandos são:

\begin{multicols}{2}
  \begin{itemize}
    \item \code{\pretitle}
    \item \code{\posttitle}
    \item \code{\preauthor}
    \item \code{\postauthor}
    \item \code{\predate}
    \item \code{\postdate}
  \end{itemize}
\end{multicols}

Os comandos podem ser modificados para atingir os resultados desejados. Por
exemplo, para um título alinhado à direta, sem serifas e com a data à esquerda
e em versaletes:

\begin{minted}[autogobble]{latex}
\pretitle{\begin{flushright}\LARGE\sffamily}
\posttitle{\par\end{flushright}\vskip 0.5em}
\predate{\begin{flushleft}\large\scshape}
\postdate{\par\end{flushleft}}
\end{minted}

\section{O pacote \code{sectsty}}

O pacote \href{https://www.ctan.org/pkg/sectsty}{\code{sectsty}} provê maneiras
de modificar a fonte dos comandos de secionamento do \LaTeX{} (como
\code{\chapter}, \code{\section}, \code{\subsection} etc.).

Carregando o pacote, ganhamos comandos como \code{\allsectionsfonts}, para
modificar a fonte de todas as seções, e \code{\sectionfont}, para realizar
modificações apenas nas fontes de seções.

Se desejarmos, por exemplo, que todas as seções de um determinado documento
fiquem à direta e sejam serifadas, podemos utilizar o seguinte comando:

\begin{minted}[autogobble]{latex}
\allsectionsfont{\sffamily\raggedleft}
\end{minted}

\section{O pacote \code{geometry}}

Mudar o tamanho da página ou margens no \LaTeX{} é uma tarefa bastante fácil
com o pacote \href{https://www.ctan.org/pkg/geometry}{\code{geometry}}. Para
mudar todas as margens para \code{2cm}, por exemplo, podemos carregar o pacote
da seguinte maneira:

\begin{minted}[autogobble]{latex}
\usepackage[margin=2cm](geometry)
\end{minted}

O pacote é extremamente versátil e possui um grande número de opções
pré-configuradas, como por exemplo \code{b5paper}, um tamanho de papel bastante
utilizado para livros e envelopes.

\section{O pacote \code{fancyhdr}}

É possível mudar o conteúdo do cabeçalho e rodapé dos documentos facilmente
utilizando o pacote \href{https://www.ctan.org/pkg/fancyhdr}{\code{fancyhdr}}:

\begin{minted}[autogobble]{latex}
\usepackage{fancyhdr}
\pagestyle{fancy}
\end{minted}

Devemos usar comandos diferentes para controlar o cabeçalho e rodapé em função
do estilo de impressão que está configurado. Vejamos dois exemplos a seguir, o
primeiro para documentos impressos em apenas um lado da folha (opção
\code{oneside}):

\begin{minted}[autogobble]{latex}
\lhead{} % reseta a parte esquerda do cabeçalho
\chead{} % reseta a parte direita do cabeçalho
\rhead{\textbf{The performance of new graduates}}
\lfoot{From: K. Grant}
\cfoot{To: Dean A. Smith}
\rfoot{\thepage}
\renewcommand{\headrulewidth}{0.4pt}
\renewcommand{\footrulewidth}{0.4pt}
\end{minted}

…e o segundo para documentos impressos em ambos os lados da folha
(\code{twoside}).

\begin{minted}[autogobble]{latex}
\fancyhead{} % limpa todos os cabeçalhos
\fancyhead[RO,LE]{\textbf{The performance of new graduates}}
\fancyfoot{} % limpa todos os rodapés
\fancyfoot[LE,RO]{\thepage} % \thepage é o número da página
\fancyfoot[LO,CE]{From: K. Grant} % LO é significa Left Odd; CE significa
                                  % Center Even; etc.
\fancyfoot[CO,RE]{To: Dean A. Smith}
\renewcommand{\headrulewidth}{0.4pt}
\renewcommand{\footrulewidth}{0.4pt}
\end{minted}

\section{O pacote \code{tcolorbox}}

O pacote \href{https://www.ctan.org/pkg/tcolorbox}{\code{tcolorbox}} permite a
criação de caixas de texto coloridas e decoradas, desde muito simples até
bastante complexas. Elas podem conter um título e podem também ser dividas em
duas partes, o que é útil para mostrar código e seu output, por exemplo.

A sintaxe é bastante simples e é bastante fácil conseguir efeitos bons com
poucas opções. O pacote é muito extensível, o que significa que ler a
documentação e encontrar exemplos é fundamental. A sintaxe básica é reproduzida
a seguir:

\begin{minted}[autogobble]{latex}
\begin{tcolorbox}[options]
…
\end{tcolorbox}
\end{minted}

Um exemplo com título, cor de fundo e dividido ao meio:

\begin{minted}[autogobble]{latex}
\begin{tcolorbox}[colback=red!5!white,colframe=red!75!black,title=Meu título]
  Este é um exemplo de \textbf{tcolorbox}.
  \tcblower
  Aqui está a parte de baixo da caixa de texto.
\end{tcolorbox}
\end{minted}

\section{O pacote \code{microtype}}

O pacote \href{https://www.ctan.org/pkg/microtype}{\code{microtype}} “provê ao
\LaTeX{} extensões como protusão de caracteres, expansão de fonte, ajuste entre
palavras e kerning adicionais.” Ele melhora consideravelmente a disposição das
fontes no documento, como ilustrado
\href{http://www.khirevich.com/latex/microtype/}{neste artigo escrito por
Siarhei Khirevich}. Não é necessário configurá-lo.

\end{document}
