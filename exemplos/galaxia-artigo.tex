%
% galaxia-artigo.tex
%
% Workshop de LaTeX do SciELO
%
% Demonstra:
% - A utilização da classe galaxia.cls
%

\documentclass[11pt,a4paper,twoside]{galaxia}
\usepackage{polyglossia}
  \setdefaultlanguage{brazil}
  \setotherlanguage{english}
\usepackage{blindtext}

\title{O cinema silencioso e o som no Brasil (1894-1920)\thanks{Uma versão
deste trabalho foi apresentada no GT “Comunicação e Cultura”, da COMPÓS de
2016. Agradeço os apontamentos dos colegas do evento e em especial de meu
supervisor de pesquisa, Prof. Eduardo Morettin, visando ao aprimoramento do
texto.}}
\author{Danielle Crepaldi Carvalho}
\issue{34}{jan--abr}{2017}
\doi{10.1590/1982-255420172808}
\pages{85}{97}

\begin{document}
\pagestyle{fancy}
\frenchspacing

\maketitle
\thispagestyle{paginaum}

\begin{abstract}{cinema silencioso; história do cinema brasileiro; som e cinema}
  \blindtext
\end{abstract}

\selectlanguage{english}
\begin{abstract}[Keywords]{silent movies; history of Brazilian cinema; sound and cinema}
  \blindtext
\end{abstract}
\selectlanguage{brazil}
\clearpage

\section{O sonoro cinema silencioso}
\Blindtext[5]

\subsection{O caso brasileiro}
\Blindtext[7]
\end{document}
