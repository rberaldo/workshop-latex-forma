%
% listas.tex
%
% Workshop de LaTeX
%
% Demonstra:
% - Três ambientes para listas: itemize, enumerate e description
% - Sintaxe para listas e listas aninhadas
% - Customização do ambiente enumerate com o pacote enumerate
%

\documentclass[a4paper,oneside]{article}
\usepackage{fontspec}
\usepackage{polyglossia}
  \setdefaultlanguage{brazil}
\usepackage{enumerate}

\begin{document}
\frenchspacing

\section{Listas}

\LaTeX{} vem com três ambientes para criar listas: \texttt{itemize, enumerate}
e \texttt{description}.

% Mudar o tipo de lista para enumerate
\begin{itemize}
  \item O ambiente \texttt{itemize} é geralmente usado para listas cuja ordem
    não é importante.
  \item A numeração que listas do tipo \texttt{enumerate} trazem pode indicar
    os passos necessários para completar uma tarefa, ou sua ordem de
    importância.
  \item A lista do tipo \texttt{description} é excelente para explicar
    conceitos relacionados. Que oportunidade perdida de usá-la!
\end{itemize}

Listas de descrição têm uma sintaxe um pouco diferente:

\begin{description}
  \item[Bit] Abreviação de \emph{binary digit}, um bit pode tomar o valor de~1
    ou~2, apenas.
  \item[Byte] Uma unidade de informação digital que geralmente tem oito bits.
    O~“y” foi escolhido de propósito para evitar que fosse acidentalmente
    confundido com “bit”.
\end{description}

\section{Customizando o ambiente \texttt{enumerate}}

É possível customizar o ambiente \texttt{enumerate} adicionando o pacote
\texttt{enumerate} ao seu preâmbulo. O resultado é que o ambiente aceitará um
argumento na forma de uma string que contém um desses valores: \texttt{A,
a, I, i} ou~\texttt{1}.

% Demonstrar diferentes possibilidades de customização
\begin{enumerate}[A)]
  \item Tales de Mileto
  \item Pitágoras
  \item Xenófanes
  \item Empédocles
  \item Aristóteles
\end{enumerate}

\end{document}
