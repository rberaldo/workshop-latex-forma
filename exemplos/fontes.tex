%
% fontes.tex
%
% Workshop de LaTeX
%
% Demonstra:
% - Inserção de diacríticos antes e hoje
% - Tamanhos e estilos de fonte
% - Como selecionar outras fontes
%

\documentclass[11pt,a4paper,oneside]{article}
\usepackage{fontspec}
% Selecionar fontes:
% \setmainfont{Linux Libertine}
%
% Selecionar a língua:
\usepackage{polyglossia}
  \setdefaultlanguage{brazil}

\title{Fontes no \LaTeX}
\author{Rafael Beraldo}

\begin{document}
\frenchspacing

\maketitle

\section{Codificações}

Este é um parágrafo cheio de acentos e palavras. Antigamente, seria necessário
escrever “tip\'{o}grafos trabalhar\~{a}o”, mas hoje é fácil adicionar símbolos
Unicode diretamente, como esta seta: →

\section{Fontes e suas famílias}

No passado, fontes não costumavam ter \emph{tantos} estilos. Tipógrafos
compunham livros inteiros com apenas um tipo e um tamanho. Hoje, temos uma
miríade de possibilidades. Empregá-las com sabedoria e, talvez, um pouco de
parcimônia não são más ideias.

As fontes que usamos comumente contam com tipos como:

\begin{itemize}
  \item O \emph{itálico,} geralmente usado para enfatizar ideias.
  \item O \textbf{negrito}, ou \textbf{bold}, muitas vezes usado para chamar a
    atenção do leitor.
  \item Os \textsc{versaletes}, que são letras em estilo de maiúscula, mas com
    a mesma altura do corpo da fonte. Uma boa ideia é usá-los em siglas, como
    \textsc{ibge}, \textsc{bc}, 3~\textsc{am} etc. Em nomes próprios e
    acrônimos geográficos, geralmente usamos maiúsculas como JRR Tolkien.
  \item Os \texttt{tipos monoespaçados} são ótimos para dar exemplos de
    código-fonte ou nomes de arquivos, como \texttt{fontes.tex}.
\end{itemize}

\section{Tamanhos de fontes}

Como já vimos, certos comandos como \verb+\section+ e \verb+chapter+ escolhem o
tamanho e espaçamento adequados para que nosso texto pareça organizado e
fluido. Essas decisões são tomadas pelos designers das classes que usamos e
baseadas em \textbf{escalas} tipográficas. Nas raras ocasiões em que precisamos
\footnotesize diminuir \Large ou aumentar \normalsize nosso texto, temos os seguintes
comandos à nossa disposição:

\begin{itemize}
  \item\verb+\tiny+
  \item\verb+\scriptsize+
  \item\verb+\footnotesize+
  \item\verb+\small+
  \item\verb+\normalsize+
  \item\verb+\large+
  \item\verb+\Large+
  \item\verb+\LARGE+
  \item\verb+\huge+
  \item\verb+\Huge+
\end{itemize}

{\Large É possível conter nosso texto entre duas chaves} para que apenas uma
seção seja afetada pelo comando de tamanho.

\end{document}
