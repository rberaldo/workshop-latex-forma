%
% linguistica.tex
%
% Workshop de LaTeX
%
% Demonstra:
% - Transcrições fonéticas com o pacote tipa
% - Árvores sintáticas usando o pacote forest
% - Exemplos e glosas com o linguex
%

\documentclass[11pt,a4paper,oneside]{article}
\usepackage{fontspec}
\usepackage{polyglossia}
  \setdefaultlanguage{brazil}
\usepackage{multicol}
% Pacotes para linguística
\usepackage{tipa,vowel,linguex}
\usepackage[linguistics]{forest}

\title{Linguística no \LaTeX}
\author{Workshop de LaTeX}

\begin{document}
\frenchspacing

\maketitle

O LaTeX traz uma série de pacotes que ajudam em tarefas de tipografia comuns
para linguistas. Veremos abaixo como escrever transcrições fonéticas com o IPA
de maneira fácil; como inserir árvores sintáticas que não são imagens e por
isso não perdem resolução ao serem impressas; e como inserir exemplos, glosas e
fazer referências a esses elementos no texto.

\section{Transcrições fonéticas}

O \verb+tipa+ permite que você escreva usando o Alfabeto Fonético Internacional
(IPA) sem ter que procurar e inserir símbolos a partir de uma lista. Tudo o que
for escrito dentro do comando \verb+\textipa{}+ será convertido
automaticamente, seguindo uma tabela de referência intuitiva. Por exemplo, para
produzir [lĩŋ.ˈgʷis.tɐs], basta digitar:

\begin{center}
  \textipa{l\~\i N.\textprimstress g\super wis.t5s}
\end{center}

Alguns exemplos de vogais orais:

\begin{center}
  \textipa{[i e E a O o u]}\\
  \textipa{[I U 5]}
\end{center}

E vogais nasais:

\begin{center}
  \textipa{\~o}\\
  \textipa{\~\textschwa}\\
  \textipa{\~5}
\end{center}

E também algumas vogais do português brasileiro:

\begin{center}
  \begin{multicols}{2}
    \textipa{S}\\
    \textipa{Z}\\
    \textipa{L}\\
    \textipa{\textltailn}\\
    \textipa{N}\\
    \textipa{\textfishhookr}\\
    \textipa{\textturnr}\\
    \textipa{\t{tS}}\\
    \textipa{\t{dZ}}
  \end{multicols}
\end{center}

Ainda é possível inserir diagramas de vogais de maneira bastante simples,
usando o pacote \verb+vowel+. O diagrama a seguir é idêntico ao que pode ser
encontrado na última versão do IPA:

\begin{center}
  \begin{vowel}
    \putcvowel[l]{i}{1}
    \putcvowel[r]{y}{1}
    \putcvowel[l]{e}{2}
    \putcvowel[r]{\o}{2}
    \putcvowel[l]{\textepsilon}{3}
    \putcvowel[r]{\oe}{3}
    \putcvowel[l]{a}{4}
    \putcvowel[r]{\textscoelig}{4}
    \putcvowel[l]{\textscripta}{5}
    \putcvowel[r]{\textturnscripta}{5}
    \putcvowel[l]{\textturnv}{6}
    \putcvowel[r]{\textopeno}{6}
    \putcvowel[l]{\textramshorns}{7}
    \putcvowel[r]{o}{7}
    \putcvowel[l]{\textturnm}{8}
    \putcvowel[r]{u}{8}
    \putcvowel[l]{\textbari}{9}
    \putcvowel[r]{\textbaru}{9}
    \putcvowel[l]{\textreve}{10}
    \putcvowel[r]{\textbaro}{10}
    \putcvowel{\textschwa}{11}
    \putcvowel[l]{\textrevepsilon}{12}
    \putcvowel[r]{\textcloserevepsilon}{12}
    \putcvowel{\textsci\ \textscy}{13}
    \putcvowel{\textupsilon}{14}
    \putcvowel{\textturna}{15}
    \putcvowel{\ae}{16}
  \end{vowel}
\end{center}

\section{Árvores sintáticas}

Existem muitos pacotes para a inserção de árvores sintáticas no LaTeX, mas
provavelmente o melhor e mais poderoso é o \verb+forest+. Para inserir uma
árvore, basta iniciar o ambiente `forest` e inserir uma sentença com os
constituintes delimitados por chaves. Ela será convertida automaticamente para
uma árvore.

\begin{forest}
  [CP [C] [IP [I] [VP [V] [NP] ] ] ]
\end{forest}

Por motivos de legibilidade do código, devemos pular linhas e indentar os
constituintes que estão em níveis mais baixos da árvore:

\begin{forest}
  [CP
    [C]
      [IP
        [I]
        [VP
          [V]
          [NP]
        ]
      ]
  ]
\end{forest}

A árvore a seguir é um exemplo com todos os nós rotulados e com os nós
terminais precedidos de linhas. Note como os nós terminais estão marcados como
constituintes dentro de outros constituintes, por exemplo em \verb+[DP [O menino]]+.
Note, ainda, que \verb+_i$+ significa que estamos entrando no modo de
matemática (\verb+$…$+) e inserindo um \emph{i} subscrito (\verb+_i+):

\begin{forest}
  [IP
    [DP [O menino$_i$]]
    [I$'$
      [I [chegou$_j$]]
      [VP
      [V$'$
        [V [t$_j$]]
        [DP [t$_i$]]
      ]
      ]
    ]
  ]
\end{forest}

É bastante usual omitirmos as linhas que levam aos nós terminais, pois algumas
pessoas as consideram redundantes. Isso é possível usando uma sintaxe
ligeiramente diferente. Note que, agora, há uma quebra de linha e apenas um
constituinte em \verb+[DP\\ O menino]+, por exemplo:

\begin{forest}
  [IP
    [DP\\ O menino$_i$]
    [I$'$
      [I\\ chegou$_j$]
      [VP
        [V$'$
          [V\\ t$_j$]
          [DP\\ t$_i$]
        ]
      ]
    ]
  ]
\end{forest}

O \verb+forest+ aceita uma série de customizações que não discutiremos no
momento. Vejamos apenas um exemplo. A árvore a seguir não tem todos os nós
rotulados e isso causa um comportamento estranho quando a árvore é disposta na
página. No entanto, podemos usar a opção \texttt{for tree = nice empty nodes}
para remediar a situação:

\begin{forest}
  for tree = nice empty nodes
  [
    [
      [a]
      [menina]
    ]
    [
      [comeu]
        [
          [o]
          [bolo]
        ]
    ]
  ]
\end{forest}

Essa opção, porque se encontra dentro do ambiente \verb+forest+, irá se aplicar
somente para a árvore em questão. Veja o manual do pacote para mais opções.

Às vezes, desejamos omitir informações irrelevantes em alguma árvore.
Explicitamos nossa decisão desenhando um triângulo sobre o constituinte que não
foi esmiuçado. Para isso, usamos a opção \verb+roof+, que se aplica a um
constituinte em nossa árvore. Veja no exemplo a seguir, onde o constituinte
\emph{chutada pelo menino} está simplificado:

\begin{forest}
  [IP
    [DP\\ a bola]
    [I$'$
      [I\\ foi]
      [VP
        [V$'$
          [V]
          [PartP [chutada pelo\\ menino, roof]]
        ]
      ]
    ]
  ]
\end{forest}

\section{Exemplos e glosas}

Outra tarefa especialmente comum para o linguista é apresentar seus dados em
formato de exemplos de sentenças, que são numeradas à esquerda. Não existe uma
implementação perfeita em LaTeX para isso, pois cada uma tem suas vantagens e
defeitos. O \verb+linguex+ é um desses pacotes com suporte a exemplos e glosas.
Vejamos alguns exemplos básicos:

\ex. Chegou o menino.

\ex. Chegou a carta.

\ex.* A chegou carta.

Sempre deixe uma linha em branco para indicar o fim de um exemplo. É possível
mostrar também subexemplos:

\ex.
  \a. Chegou o menino.
  \b. O menino chegou.

Também é possível escrever glosas usando o comando \verb+\exg.+. As palavras
serão alinhadas automaticamente:

\exg.
  Gila abur-u-n ferma hamišaluǧ güǧüna amuq’-da-č.\\
  now they-OBL-GEN farm forever behind stay-FUT-NEG\\
  ``Now their farm will not stay behind forever.''

\subsection{Referências cruzadas}

Já vimos que os comandos \verb+\label+ e \verb+\ref+ permitem criar referências
que são automaticamente numeradas e atualizadas pelo LaTeX. Esses comandos
também funcionam nos exemplos e glosas do \verb+linguex+:

\ex.\label{ex:gram} Chegou o menino.

\ex.*\label{ex:agram} A chegou carta.

Em~\ref{ex:gram}, temos uma sentença bem formada, ao contrário
de~\ref{ex:agram}, que é uma sentença agramatical.

\end{document}
