%
% receita.tex
%
% Workshop de LaTeX do SciELO
%
% Problema: a receita a seguir está incompleta. Adicione os ingredientes do
% slide usando o ambiente enumerate e customizando-o de acordo com as
% instruções.
%

\documentclass[a4paper,oneside]{article}
\usepackage{fontspec}
\usepackage{polyglossia}
  \setdefaultlanguage{brazil}
\usepackage{enumerate}

\title{Panqueca: o Hello, World! das receitas\footnote{Adaptado do livro
\emph{Cooking for Geeks}, primeira edição, escrito por Jeff Potter.}}
\author{Jeff Potter}
\date{Julho de 2010}

\begin{document}
\frenchspacing

\maketitle

Em uma tigela, pese e misture:

% Lista de ingredientes 1:
% - 190g de farinha
% - 25g de açúcar
% - 10g de fermento químico em pó
% - 3g de sal

Em uma tigela separada e que possa ir ao micro-ondas, derreta:

% Lista de ingredientes 2:
% - 25g de manteiga

Adicione à manteiga e bata para incorporar completamente:

% Lista de ingredientes 3:
% - 330g de leite
% - 80g de ovos

Derrame os ingredientes sólidos sobre os líquidos e bata até que os
ingredientes estejam quase que completamente incorporados. Bolinhas de farinha
não são um problema: é melhor evitar bater a massa demais, para minimizar a
formação de glúten a partir de duas proteínas, a glutenina e a gliadina, que
estão presentes na farinha (elas se ligam formando uma matriz parecida com uma
rede).

Coloque uma frigideira não-aderente no fogo médio-alto. Espere até que a panela
esteja quente. O teste padrão é jogar algumas gotas de água na panela e
observar se chiam; o teste geek é usar um termômetro infravermelho e checar se
a panela chegou a 200~\textdegree~C. Usando uma concha, copo de medida ou
colher de sorvete, despeje cerca de meio copo de massa na panela. Enquanto o
primeiro lado cozinha, é possível observar bolhas se formando na superfície de
cima da panqueca. Vire a panqueca após que as bolham comecem a se formar, mas
antes que estourem (cerca de dois minutos).
\end{document}
