% Escrevendo classes %%%%%%%%%%%%%%
\section{Escrevendo classes}

% Podemos reunir as modificações que fizemos anteriormente em um arquivo .cls
\begin{frame}
  \frametitle{Escrevendo classes}
  \LARGE
  O que é uma classe (\filename{cls})?
\end{frame}

% Alguns exemplos de classes
\begin{frame}
  \frametitle{Escrevendo classes}
  \LARGE
  \latexcode{article, book, memoir, KOMA, …}
\end{frame}

% Comandos para autores de classes são em letras maiúsculas
\begin{frame}
  \frametitle{Escrevendo classes: comandos}
  \LARGE
  \latexcode{\RequirePackage} ao invés de \latexcode{\usepackage}
\end{frame}

% Partes de uma classe
\begin{frame}
  \frametitle{Escrevendo classes: estrutura}
  \LARGE
  Em geral, são escritas na ordem:
  \vspace{1em}

  \begin{itemize}
    \item Identificação
    \item Declarações preliminares
    \item Opções
    \item Mais declarações
  \end{itemize}
\end{frame}

% Exemplos
\begin{frame}
  \frametitle{Escrevendo classes: exemplos}
  \LARGE
  Vejamos \filename{exemplos/colorida.cls}
\end{frame}

% Exemplos
\begin{frame}
  \frametitle{Escrevendo classes: exemplos}
  \LARGE
  Vejamos \filename{exemplos/galaxia.cls}
\end{frame}

% Exemplos
\begin{frame}
  \frametitle{Escrevendo classes: exemplos}
  \LARGE
  Vejamos \href{https://github.com/RocketshipGames/gapd.els/}{\code{gapd}}
  (GitHub)
\end{frame}

% Exemplos
\begin{frame}
  \frametitle{Escrevendo classes: exemplos}
  \LARGE
  Vejamos
  \href{https://github.com/noaham/research_note_cls/}{\code{research\_note}}
  (GitHub)
\end{frame}

% Exercício
\begin{frame}
  \frametitle{Escrevendo classes: exercício}
  \LARGE
  Resolver \filename{exercicios/estomatoher.cls}
\end{frame}
