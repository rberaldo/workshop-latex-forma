% Matemática %%%%%%%%%%%%%%
\section{Matemática}

% Até agora, estávamos no modo de texto. No modo de matemática, o modo como o
% LaTeX interpreta o que escrever é diferente. Além disso, o modo de matemática
% é dividido em dois tipos: inline e displayed
\begin{frame}
  \frametitle{Matemática: modos do \LaTeX}
  \LARGE
  \only<1>{Modo de texto vs.\\ modo de matemática}
  \only<2>{Modo de matemática:\\ \emph{inline} e \emph{displayed}}
\end{frame}

% Existem três ambientes para acessar o modo de matemática.
\begin{frame}[fragile]
  \frametitle{Matemática: ambientes}
  \LARGE
  Três ambientes:\\

  \only<1>{\latexcode{math} ou \latexcode{\( … \)}}
  \only<2>{\latexcode{displaymath} ou \latexcode{\[ … \]}}
  \only<3>{\latexcode{equation}}
\end{frame}

% Há uma infinidade de comandos, pacotes e técnicas para aprender. Cobriremos o
% básico.
\begin{frame}
  \frametitle{Matemática}
  \LARGE
  Cobriremos o básico!

  Mais em \url{www.en.wikibooks.org/wiki/LaTEX/Mathematics}
\end{frame}

% Símbolos em modo matemático
\begin{frame}[fragile]
  \frametitle{Matemática: símbolos}
  \LARGE
  \latexcode{2 \times 2 = 4}
  \hfill
  \( 2 \times 2 = 4 \)
\end{frame}

% Alfabeto grego
\begin{frame}[fragile]
  \frametitle{Matemática: alfabeta grego}
  \LARGE
  \latexcode{\alpha, \beta, \pi}
  \hfill
  \( \alpha, \beta, \pi \)
\end{frame}

% Operadores
\begin{frame}[fragile]
  \frametitle{Matemática: operadores}
  \begin{minted}[autogobble,fontsize=\Large,breaklines]{latex}
    \cos (2\theta) = \cos^2 \theta - \sin^2 \theta
  \end{minted}

  \LARGE
  \[ \cos (2\theta) = \cos^2 \theta - \sin^2 \theta \]
\end{frame}

% Potências e subscritos
\begin{frame}[fragile]
  \frametitle{Matemática: potências e subscritos}
  \LARGE

  \begin{tabular}{r|l}
    \latexcode{2^8}             & \( 2^8 \)\\
    \latexcode{a_b}             & \( a_b \)\\
    \latexcode{2^{32}}          & \( 2^{32} \)\\
    \latexcode{f(n) = 4n + n^2} & \( f(n) = 4n + n^2 \)
  \end{tabular}
\end{frame}

% Frações
\begin{frame}[fragile]
  \frametitle{Matemática: frações}
  \LARGE
  \begin{minipage}{.45\textwidth}
    \begin{minted}[autogobble,fontsize=\large,breaklines]{latex}
    F = G \frac{m_1 m_2}{d^2}

    \frac{\frac{1}{x} + \frac{1}{y}}{y-z}
    \end{minted}
  \end{minipage}
  \hfill
  \begin{minipage}{.45\textwidth}
    \[ F = G \frac{m_1 m_2}{d^2} \]

    \[ \frac{\frac{1}{x}+\frac{1}{y}}{y-z} \]
  \end{minipage}
\end{frame}

% Raízes
\begin{frame}[fragile]
  \frametitle{Matemática: raízes}
  \LARGE
  \begin{tabular}{r|l}
    \latexcode{\sqrt{10^2} = 10}      & \( \sqrt{10^2} = 10 \) \\
    \latexcode{\sqrt[3]{\frac{a}{b}}} & \( \sqrt[3]{\frac{a}{b}} \)
  \end{tabular}
\end{frame}

% Estudar mais exemplos de matemática
\begin{frame}
  \frametitle{Matemática: exemplo}
  \Huge
  Estudar \filename{exemplos/\\matematica.tex}
\end{frame}

% Exercício: reproduza essa equação em equacao.tex
\begin{frame}
  \frametitle{Matemática: exercício}
  \LARGE
  Reproduza em \filename{exercicios/equacao.tex}:

  \begin{equation}
    x = \frac{-b \pm \sqrt{b^2 - 4ac}}{2a}
  \end{equation}
\end{frame}
