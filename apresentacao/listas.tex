% Listas %%%%%%%%%%%%%%
\section{Listas}

% Três tipos de listas inclusos por padrão
\begin{frame}
  \frametitle{Listas: três tipos}
  \LARGE
  Ambientes: \code{itemize, enumerate} e \code{description}
\end{frame}

% Anatomia de uma lista
\begin{frame}[fragile]
  \frametitle{Listas: sintaxe}
  \LARGE
  Ingredientes para carbonara:

  \begin{minted}[autogobble,fontsize=\LARGE,breaklines]{latex}
    \begin{itemize}
      \item Bacon
      \item Macarrão
      \item Ovos
      \item Parmesão
      \item Pimenta-do-reino
    \end{itemize}
  \end{minted}
\end{frame}

% Existem mais listas no arquivo exemplos/listas.tex
\begin{frame}
  \frametitle{Listas: exemplo}
  \Huge
  Aprenderemos mais em \filename{exemplos/listas.tex}
\end{frame}

% Exercício: completar a receita de panqueca com os ingredientes corretos.
\begin{frame}
  \frametitle{Listas: exercício}
  \Huge
  Resolver \filename{exercicios/receita.tex}
\end{frame}

% Lista de ingredientes que os participantes devem colocar na receita. Notar a
% palavra ingrediente, o número sequencial e o parênteses.
\begin{frame}
  \frametitle{Listas: exercício}
  \large
  \begin{enumerate}[{Ingrediente} 1)]
    \item 190g de farinha
    \item 25g de açúcar
    \item 10g de fermento químico em pó
    \item 3g de sal\\[1em] … texto …\\[1em]
    \item 25g de manteiga\\[1em] … texto …\\[1em]
    \item 330g de leite
    \item 80g de ovos
  \end{enumerate}
\end{frame}
