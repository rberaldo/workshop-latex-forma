% Macros %%%%%%%%%%%%%%
\section{Macros}

% Uma das maiores vantagens do LaTeX é sua extensibilidade
\begin{frame}
  \frametitle{Macros}
  \LARGE
  \LaTeX{} é extensível
\end{frame}

% O LaTeX é um conjunto de macros para TeX
\begin{frame}
  \frametitle{Macros}
  \LARGE
  Afinal, \LaTeX{} é um conjunto\\ de macros para o \TeX
\end{frame}

% Macros são programas que automatizam certas funções
\begin{frame}
  \frametitle{Macros}
  \LARGE
  Macros automatizam funções
\end{frame}

% Macros de substituição
\begin{frame}[fragile]
  \frametitle{Macros de substituição}
  \LARGE
  \latexcode{\newcommand{\forma}{ForMA}}
  \vspace{1em}

  Workshop de LaTeX no \latexcode{\forma{}} em setembro.
\end{frame}

% xspace
\begin{frame}[fragile]
  \frametitle{Macros de substituição: o pacote \code{xspace}}
  \begin{minted}[autogobble,fontsize=\large,breaklines]{latex}
  \usepackage{xspace}
  …
  \newcommand{\forma}{ForMA\xspace}
  …
  Workshop de LaTeX no \forma em setembro.
  \end{minted}
\end{frame}

% Macros com variáveis
\begin{frame}
  \frametitle{Macros com variáveis}
  \LARGE
  Macros como \latexcode{\maketitle} usam variáveis como
  \latexcode{\@author}
\end{frame}

% Comandos com @ podem ser acessados após transformar @ em uma letra
\begin{frame}
  \frametitle{Macros com variáveis: acessando comandos reservados}
  \LARGE
  \latexcode{\makeatletter … \makeatother}
\end{frame}

% Veremos um exemplo de como isso pode ser usado para redefinir o comando
% \maketitle
\begin{frame}[fragile]
  \frametitle{Macros com variáveis: customizando o
  \latexcode{\maketitle}}
  \LARGE
  Veremos como customizar o \latexcode{\maketitle} usando o comando:

  \begin{minted}[autogobble,fontsize=\LARGE,breaklines]{latex}
  \renewcommand{\maketitle}{…}
  \end{minted}
\end{frame}

% Macros com argumentos
\begin{frame}[fragile]
  \frametitle{Macros com argumentos}
  \LARGE
  Macros podem levar argumentos:

  \begin{minted}[autogobble,fontsize=\LARGE,breaklines]{latex}
  \textbf{texto}
  \end{minted}
\end{frame}

% Sintaxe de macros com argumentos
\begin{frame}[fragile]
  \frametitle{Macros com argumentos: sintaxe}
  \begin{minted}[autogobble,fontsize=\LARGE,breaklines]{latex}
  \newcommand{\eng}[1]{%
    \emph{\textenglish{#1}}%
  }
  …
  \eng{some text in English}
  \end{minted}
\end{frame}

% Também é possível definir novos ambientes
\begin{frame}[fragile]
  \frametitle{Novos ambientes}
  \begin{minted}[autogobble,fontsize=\LARGE,breaklines]{latex}
  \newenvironment{italics}
  {\itshape}
  {}
  \end{minted}
\end{frame}

% Exemplos de macros e redefinição do título
\begin{frame}
  \frametitle{Macros: exemplos}
  \huge
  Estudar \filename{exe\-mplos/ma\-cros.tex}
\end{frame}

% Vamos aprender a utilizar macros
\begin{frame}
  \frametitle{Macros: exercício}
  \huge
  Resolver \filename{exercicio/auto\-ma\-ti\-zan\-do.tex}
\end{frame}

\begin{frame}
  \frametitle{Macros: exercício: \latexcode{\address} e \latexcode{\telephone}}
  \LARGE
  Escrever duas macros, \latexcode{\address} e \latexcode{\telephone}, que
  expandam para o endereço e telefone do ForMA
\end{frame}

\begin{frame}
  \frametitle{Macros: exercício: \latexcode{\email} e \latexcode{\todo}}
  \LARGE
  Resultado de \latexcode{\email}: \email{rberaldo@cabaladada.org}
  \vspace{1em}

  Resultado de \latexcode{\todo}: \todo{texto usando o \latexcode{todo}}
\end{frame}

\begin{frame}
  \frametitle{Macros: exercício: mudar o \latexcode{\maketitle}}
  \LARGE
  Vamos ao documento mudar o \latexcode{\maketitle} juntos
\end{frame}
