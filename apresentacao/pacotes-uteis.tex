% Pacotes úteis %%%%%%%%%%%%%%
\section{Pacotes úteis}

% titling
\begin{frame}
  \frametitle{Pacotes úteis: \latexcode{titling}}
  \LARGE
  Podemos modificar o \latexcode{\maketitle} manualmente ou utilizar um pacote
  como \latexcode{titling}
\end{frame}

\begin{frame}
  \frametitle{Pacotes úteis: \latexcode{titling}}
  \LARGE
  Provê comandos como \latexcode{\pretitle} e \latexcode{\posttitle}, bem como
  \latexcode{\thetitle}
\end{frame}

% sectsty
\begin{frame}
  \frametitle{Pacotes úteis: \latexcode{sectsty}}
  \LARGE
  Permite controlar o estilo dos comandos de secionamento com comandos como
  \latexcode{\allsectionfonts} e \latexcode{\sectionfont}
\end{frame}

% titlesec
\begin{frame}
  \frametitle{Pacotes úteis: \latexcode{titlesec}}
  \LARGE
  Mais complexo e poderoso que o \latexcode{sectsty}
\end{frame}

% fancyhdr
\begin{frame}
  \frametitle{Pacotes úteis: \latexcode{fancyhdr}}
  \LARGE
  Permite mudar o conteúdo do cabeçalho e rodapé
\end{frame}

% geometry
\begin{frame}
  \frametitle{Pacotes úteis: \latexcode{geometry}}
  \LARGE
  Provê controle fácil do tamanho da página e margens
\end{frame}

% tcolorbox
\begin{frame}
  \frametitle{Pacotes úteis: \latexcode{tcolorbox}}
  \LARGE
  Ambiente para a criação de caixas de texto, como por exemplo:
  \vfill

  \begin{tcolorbox}[title=Caixa de texto]
    Uma caixa de texto…
    \tcblower
    …dividida ao meio.
  \end{tcolorbox}
\end{frame}

% microtype
\begin{frame}
  \frametitle{Pacotes úteis: \latexcode{microtype}}
  \LARGE
  \begin{quote}
  Provê extensões como protusão de caracteres, expansão de fonte, ajuste entre
  palavras e kerning adicionais.\par\hfill (Manual do \latexcode{microtype})
  \end{quote}
\end{frame}

% Exemplos
\begin{frame}
  \frametitle{Pacotes úteis: exemplos}
  \LARGE
  Vejamos \filename{exemplos/pacotes-uteis.tex} e
  \filename{exemplos/cinema-silencioso.tex}
\end{frame}

% Exercício
\begin{frame}
  \frametitle{Pacotes úteis: exercício}
  \LARGE
  Resolver \filename{exercicios/estomatol\-herediana.pdf}
\end{frame}

% Escrevendo pacotes
\begin{frame}
  \frametitle{Escrevendo pacotes}
  \LARGE
  É possível escrever pacotes (\latexcode{sty})
\end{frame}
