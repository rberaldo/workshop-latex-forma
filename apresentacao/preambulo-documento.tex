% Preambulo do documento %%%%%%%%%%%%%%
\section{Preâmbulo do documento}

% Dar uma olhada no arquivo. Ensinar a distinção entre o preâmbulo e o corpo do
% documento.
\begin{frame}
  \frametitle{Preâmbulo do documento}
  \LARGE
  Documentos \LaTeX: preâmbulo e corpo
\end{frame}

\begin{frame}[fragile]
  \frametitle{Preâmbulo do documento}
  \LARGE
  \begin{minted}[autogobble,fontsize=\large,breaklines]{latex}
  \documentclass[11pt,a4paper,oneside]{article}
  \end{minted}
\end{frame}

% Explicar as opções da classe article que escolhi
\begin{frame}[fragile]
  \frametitle{Preâmbulo do documento}
  \LARGE
  Classes padrão:
  \begin{multicols}{2}
    \begin{itemize}
      \item\code{article}
      \item\code{report}
      \item\code{book}
      \item\code{letter}
      \item\code{memoir}
      \item\code{beamer}
  \end{itemize}
\end{multicols}
\end{frame}

% Vejamos quais são as opções de classe mais comuns
\begin{frame}
  \frametitle{Preâmbulo do documento}
  \LARGE
  Opções de classe comuns:
  \begin{itemize}
    \only<1>{\item \code{10pt, 11pt, 12pt}}
    \only<1>{\item \code{a4paper, a5paper, letterpaper, …}}
    \only<2>{\item \code{titlepage, notitlepage}}
    \only<2>{\item \code{twocolumn}}
    \only<2>{\item \code{twoside, oneside}}
    \only<3>{\item \code{landscape}}
    \only<3>{\item \code{openright, openany}}
    \only<3>{\item \code{draft}}
  \end{itemize}
\end{frame}

\begin{frame}
  \frametitle{Preâmbulo do documento}
  \huge
  Vejamos \filename{exemplos/artigo.tex}
\end{frame}
