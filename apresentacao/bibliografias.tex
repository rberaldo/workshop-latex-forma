% Bibliografias %%%%%%%%%%%%%%
\section{Bibliografias}

% O BibTeX tem dois arquivos principais
\begin{frame}
  \frametitle{Bibliografias com o \hologo{BibTeX}: arquivos}
  \LARGE
  \hologo{BibTeX}: database (\code{bib})\\
  e estilo (\code{bst})
\end{frame}

% Um exemplo de entrada bibliográfica. Explicar o uso de chaves em certos
% lugares.
\begin{frame}[fragile]
  \frametitle{Bibliografias com o \hologo{BibTeX}: exemplo}
  \LARGE
  Exemplo de um arquivo \code{.bib}:
  \vspace{1em}

  \begin{minted}[autogobble,fontsize=\large,breaklines]{bibtex}
    @article{greenwade93,
      author  = "George D. Greenwade",
      title   = "The {C}omprehensive {T}ex {A}rchive {N}etwork ({CTAN})",
      year    = "1993",
      journal = "TUGBoat",
      volume  = "14",
      number  = "3",
      pages   = "342--351"
    }
  \end{minted}
\end{frame}

% No local desejado, colocamos a bibliografia
\begin{frame}[fragile]
  \frametitle{Bibliografias com o \hologo{BibTeX}: inserir arquivo \code{.bib}}
  \LARGE
  \latexcode{\bibliography{arquivo}}
\end{frame}

% Para citar, basta usar um desses comandos
\begin{frame}[fragile]
  \frametitle{Bibliografias com o \hologo{BibTeX}: como citar no texto}
  \begin{minted}[autogobble,fontsize=\LARGE,breaklines]{latex}
  \cite[p.~20]{greenwade93}
  \citeonline[p.~20]{greenwade93}
  \end{minted}
\end{frame}

% Mais um live coding!
\begin{frame}
  \frametitle{Bibliografias com o \hologo{BibTeX}: exemplo}
  \huge
  Estudar \filename{exemplos/abntex2/\\trabalho-normatizado.tex}
\end{frame}
