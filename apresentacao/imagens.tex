% Imagens %%%%%%%%%%%%%%
\section{Imagens}

% Para inserir imagens, precisamos carregar o pacote graphicx
\begin{frame}
  \frametitle{Imagens}
  \LARGE
  Pacote \code{graphicx}
\end{frame}

% O pacote graphics nos dá acesso ao comando \includegraphics, que aceita uma
% série de opções. Não discutiremos todas em nosso workshop, mas as mais úteis
% são width, height, scale e keepaspectratio.
\begin{frame}[fragile]
  \frametitle{Imagens: como carregar gráficos}
  \Large
  \latexcode{\includegraphics[opções]{imagem}}
  \vspace{1em}

  Algumas opções:
  \begin{itemize}
    \item \code{width} e \code{height}
    \item \code{scale}
    \item \code{keepaspectratio} (bool)
  \end{itemize}
\end{frame}

% Geralmente, usamos o comando \includegraphics com o ambiente figure, que é
% bastante parecido com o ambiente table, que acabamos de estudar.
\begin{frame}[fragile]
  \frametitle{Imagens}
  \Large
  Ambiente \code{figure}:

    \begin{minted}[autogobble,fontsize=\Large,breaklines]{latex}
    \begin{figure}[h]
      \centering
      \includegraphics{imagem}
      \caption{Exemplo de imagem}
      \label{fig:imagem}
    \end{figure}
    \end{minted}
\end{frame}

% Demonstrar os conceitos.
\begin{frame}
  \frametitle{Imagens}
  \Huge
  Estudar \filename{exemplos/imagens.tex}
\end{frame}

% Exercício: colocar uma figura na tabela que fizemos anteriormente.
\begin{frame}
  \frametitle{Imagens}
  \Huge
  Resolver \filename{exercicios/\\ilustrado.tex}
\end{frame}
