% Pacotes úteis para linguistas %%%%%%%%%%%%%%
\section{Linguística}

% tipa
% baseado em https://social.stoa.usp.br/articles/0029/0658/APRES-tipa.pdf
\begin{frame}
  \frametitle{Linguística: transcrições com o \latexcode{tipa}}
  \huge
  O IPA é facilmente acessível\\ usando o pacote \latexcode{tipa}
\end{frame}

% \textipa{}, {\tipaencoding} e ambiente IPA
\begin{frame}[fragile]
  \frametitle{Linguística: \latexcode{tipa}}
  \huge

  Três modos de acessar:
  \begin{onlyenv}<1>
    \begin{minted}[autogobble,breaklines]{latex}
      \textipa{…}
    \end{minted}
  \end{onlyenv}

  \begin{onlyenv}<2>
    \begin{minted}[autogobble,breaklines]{latex}
      {\tipaencoding …}
    \end{minted}
  \end{onlyenv}

  \begin{onlyenv}<3>
    o ambiente \latexcode{IPA}
  \end{onlyenv}
\end{frame}

% Exemplos básicos
\begin{frame}[fragile]
  \frametitle{Linguística: exemplo com o \latexcode{tipa}}
  \huge

  \begin{minted}[autogobble,fontsize=\large,breaklines]{latex}
    \textipa{l\~\i N.\textprimstress g\super wis.t5s}
  \end{minted}

  \textipa{l\~\i N.\textprimstress g\super wis.t5s}
\end{frame}

% Vogais orais do ptbr
\begin{frame}[fragile]
  \frametitle{Linguística: vogais orais com o \latexcode{tipa}}
  \huge

  \begin{minted}[autogobble,fontsize=\Large,breaklines]{latex}
    \textipa{[i e E a O o u]}

    \textipa{[I U 5]}
  \end{minted}

  \textipa{[i e E a O o u]}

  \textipa{[I U 5]}
\end{frame}

% Vogais nasais
\begin{frame}[fragile]
  \frametitle{Linguística: vogais nasais com o \latexcode{tipa}}
  \Large
  \setlength{\tabcolsep}{12pt}
  \begin{tabular}{rl}
    \latexcode{\textipa{\~o}}          & \textipa{\~o} \\
    \latexcode{\textipa{\~\textschwa}} & \textipa{\~\textschwa} \\
    \latexcode{\textipa{\~5}}          & \textipa{\~5}
  \end{tabular}
\end{frame}

% Consoantes
\begin{frame}
  \frametitle{Linguística: consoantes com o \latexcode{tipa}}
  \Large
  \setlength{\tabcolsep}{12pt}
  \begin{tabular}{rl}
    \latexcode{\textipa{S}}              & \textipa{S} \\
    \latexcode{\textipa{Z}}              & \textipa{Z} \\
    \latexcode{\textipa{L}}              & \textipa{L} \\
    \latexcode{\textipa{\textltailn}}    & \textipa{ \textltailn} \\
    \latexcode{\textipa{N}}              & \textipa{N} \\
    \latexcode{\textipa{\textfishhookr}} & \textipa{\textfishhookr} \\
    \latexcode{\textipa{\textturnr}}     & \textipa{\textturnr} \\
    \latexcode{\textipa{\t{tS}}}         & \textipa{\t{tS}} \\
    \latexcode{\textipa{\t{dZ}}}         & \textipa{\t{dZ}}
  \end{tabular}
\end{frame}

% - vowel chart
\begin{frame}[fragile]
  \frametitle{Linguística: diagrama de vogais com o \latexcode{tipa}}
  \large
  \begin{minipage}{.45\textwidth}
    \begin{minted}[autogobble,fontsize=\tiny,breaklines]{latex}
      \begin{vowel}
        \putcvowel[l]{i}{1}
        \putcvowel[r]{y}{1}
        \putcvowel[l]{e}{2}
        \putcvowel[r]{\o}{2}
        …
        \putcvowel{\textsci\ \textscy}{13}
        \putcvowel{\textupsilon}{14}
        \putcvowel{\textturna}{15}
        \putcvowel{\ae}{16}
      \end{vowel}
    \end{minted}
  \end{minipage}
  \hfill
  \begin{minipage}{.45\textwidth}
    \begin{vowel}
      \putcvowel[l]{i}{1}
      \putcvowel[r]{y}{1}
      \putcvowel[l]{e}{2}
      \putcvowel[r]{\o}{2}
      \putcvowel[l]{\textepsilon}{3}
      \putcvowel[r]{\oe}{3}
      \putcvowel[l]{a}{4}
      \putcvowel[r]{\textscoelig}{4}
      \putcvowel[l]{\textscripta}{5}
      \putcvowel[r]{\textturnscripta}{5}
      \putcvowel[l]{\textturnv}{6}
      \putcvowel[r]{\textopeno}{6}
      \putcvowel[l]{\textramshorns}{7}
      \putcvowel[r]{o}{7}
      \putcvowel[l]{\textturnm}{8}
      \putcvowel[r]{u}{8}
      \putcvowel[l]{\textbari}{9}
      \putcvowel[r]{\textbaru}{9}
      \putcvowel[l]{\textreve}{10}
      \putcvowel[r]{\textbaro}{10}
      \putcvowel{\textschwa}{11}
      \putcvowel[l]{\textrevepsilon}{12}
      \putcvowel[r]{\textcloserevepsilon}{12}
      \putcvowel{\textsci\ \textscy}{13}
      \putcvowel{\textupsilon}{14}
      \putcvowel{\textturna}{15}
      \putcvowel{\ae}{16}
    \end{vowel}
  \end{minipage}
\end{frame}

% Referência para o tipa
\begin{frame}
  \frametitle{Linguística: referência do \latexcode{tipa}}
  \huge
  Veja \filename{exemplos/tipachart.pdf} para um resumo de como acessar os
  caracteres do~IPA.
\end{frame}

% forest
\begin{frame}
  \frametitle{Linguística: árvores sintáticas com o \latexcode{forest}}
  \huge
  Também é possível desenhar árvores sintáticas usando o pacote
  \latexcode{forest}.
\end{frame}

% Árvore simples
\begin{frame}[fragile]
  \frametitle{Linguística: árvores simples com o \latexcode{forest}}
  \Large
  \begin{columns}
    \begin{column}{.45\textwidth}
      \begin{minted}[autogobble,fontsize=\large,showspaces,breaklines]{latex}
        \begin{forest}
          [CP
            [C]
              [IP
                [I]
                [VP 
                  [V]
                  [NP]
                ]
              ]
          ]
        \end{forest}
      \end{minted}
    \end{column}
    \hfill
    \begin{column}{.45\textwidth}
      \begin{forest}
        [CP
          [C]
            [IP
              [I]
              [VP 
                [V]
                [NP]
              ]
            ]
        ]
      \end{forest}
    \end{column}
  \end{columns}
\end{frame}

% Terminais com e sem linhas
\begin{frame}[fragile]
  \frametitle{Linguística: árvores com o \latexcode{forest}}
  \large
  \begin{columns}
    \begin{column}{.45\textwidth}
      \begin{minted}[autogobble,fontsize=\small,breaklines]{latex}
        \begin{forest}
          [IP
            [DP [O menino$_i$]]
            [I$'$
              [I [chegou$_j$]]
              [VP
                [V$'$
                  [V [t$_j$]]
                  [DP [t$_i$]]
                ]
              ]
            ]
          ]
        \end{forest}
      \end{minted}
    \end{column}
    \hfill
    \begin{column}{.45\textwidth}
      \begin{forest}
        [IP
          [DP [O menino$_i$]]
          [I$'$
            [I [chegou$_j$]]
            [VP
              [V$'$
                [V [t$_j$]]
                [DP [t$_i$]]
              ]
            ]
          ]
        ]
      \end{forest}
    \end{column}
  \end{columns}
\end{frame}

\begin{frame}[fragile]
  \frametitle{Linguística: árvores com o \latexcode{forest}}
  \large
  \begin{columns}
    \begin{column}{.45\textwidth}
      \begin{minted}[autogobble,fontsize=\small,breaklines]{latex}
        \begin{forest}
          [IP
            [DP [O menino$_i$]]
            [I$'$
              [I [chegou$_j$]]
              [VP
                [V$'$
                  [V [t$_j$]]
                  [DP [t$_i$]]
                ]
              ]
            ]
          ]
        \end{forest}
      \end{minted}
    \end{column}
    \begin{column}{.5\textwidth}
      \begin{forest}
        [IP
          [DP\\ O menino$_i$]
          [I$'$
            [I\\ chegou$_j$]
            [VP
              [V$'$
                [V\\ t$_j$]
                [DP\\ t$_i$]
              ]
            ]
          ]
        ]
      \end{forest}
    \end{column}
  \end{columns}
\end{frame}

% sem rótulos
\begin{frame}[fragile]
  \frametitle{Linguística: árvores com o \latexcode{forest}}
  \small
  \begin{columns}
    \begin{column}{.5\textwidth}
      \begin{minted}[autogobble,fontsize=\footnotesize,breaklines]{latex}
        \begin{forest}
          for tree=nice empty nodes
          [
            [
              [a]
              [menina]
            ]
            [
              [comeu]
                [
                  [o]
                  [bolo]
                ]
            ]
          ]
        \end{forest}
      \end{minted}
    \end{column}
    \begin{column}{.5\textwidth}
      \begin{forest}
        for tree=nice empty nodes
        [
          [
            [a]
            [menina]
          ]
          [
            [comeu]
              [
                [o]
                [bolo]
              ]
          ]
        ]
      \end{forest}
    \end{column}
  \end{columns}
\end{frame}

% Triângulos para omitir detalhes
\begin{frame}[fragile]
  \frametitle{Linguística: árvores com o \latexcode{forest}}
  \small
  \begin{columns}
    \begin{column}{.5\textwidth}
      \begin{minted}[autogobble,fontsize=\scriptsize,breaklines]{latex}
        \begin{forest}
          [IP
            [DP\\ a bola]
            [I$'$
              [I\\ foi]
              [VP
                [V$'$
                  [V]
                  [PartP [chutada pelo menino, roof]]
                ]
              ]
            ]
          ]
        \end{forest}
      \end{minted}
    \end{column}
    \begin{column}{.5\textwidth}
      \begin{forest}
        [IP
          [DP\\ a bola]
          [I$'$
            [I\\ foi]
            [VP
              [V$'$
                [V]
                [PartP [chutada pelo\\ menino, roof]]
              ]
            ]
          ]
        ]
      \end{forest}
    \end{column}
  \end{columns}
\end{frame}

% linguex
\begin{frame}
  \frametitle{Linguística: exemplos com o \latexcode{linguex}}
  \huge
  Exemplos e glosas são facilmente representados usando o pacote
  \latexcode{linguex}.
\end{frame}

% Exemplos numerados
\begin{frame}[fragile]
  \frametitle{Linguística: exemplos com o \latexcode{linguex}}
  \large
  \begin{columns}
    \begin{column}{.5\textwidth}
      \begin{minted}[autogobble,breaklines]{latex}
        \ex. Chegou o menino.

        \ex. Chegou a carta.

        \ex.* A chegou carta.
      \end{minted}
    \end{column} 
    \begin{column}{.5\textwidth}
      \ex. Chegou o menino.\par
      \ex. Chegou a carta.\par
      \ex.* A chegou carta.\par
    \end{column} 
  \end{columns}
\end{frame}

% Glosas (tirada de https://www.eva.mpg.de/lingua/resources/glossing-rules.php)
\begin{frame}[fragile]
  \frametitle{Linguística: exemplos com o \latexcode{linguex}}
  \small

  \begin{minted}[autogobble,breaklines]{latex}
    \exg.
    Gila abur-u-n ferma hamišaluǧ güǧüna amuq’-da-č.\\
    now they-OBL-GEN farm forever behind stay-FUT-NEG\\
    ``Now their farm will not stay behind forever.''
  \end{minted}

  \exg.
  Gila abur-u-n ferma hamišaluǧ güǧüna amuq’-da-č.\\
  now they-OBL-GEN farm forever behind stay-FUT-NEG\\
  ``Now their farm will not stay behind forever.''

\end{frame}

% Referencias cruzadas
\begin{frame}[fragile]
  \frametitle{Linguística: exemplos com o \latexcode{linguex}}
  \normalsize

  \begin{minted}[autogobble,fontsize=\small,breaklines]{latex}
    \ex.\label{ex:gram} Chegou o menino.

    \ex.*\label{ex:agram} A chegou carta.

    Em~\ref{ex:gram}, temos uma sentença bem formada, ao contrário de~\ref{ex:agram}, que é uma sentença agramatical.
  \end{minted}

  \vfill

  \ex.\label{ex:gram} Chegou o menino.

  \ex.*\label{ex:agram} A chegou carta.

  Em~\ref{ex:gram}, temos uma sentença bem formada, ao contrário
  de~\ref{ex:agram}, que é uma sentença agramatical.

\end{frame}
